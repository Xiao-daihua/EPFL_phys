
\section{EEP from gravity to Geometry}

\subsection{章节内容 take home message}
本章之中,我们根据Einstein Equvalence Principle 我们知道了“加速参考系”和“生活在引力之中”的物体的参考系是有着相似性的!!并且对于有引力的体系,我们可以找到一个参考系让情况locally和没有引力是一样的【这个仅仅是locally,因为弯曲时空就是不一样的】

所以我们为了研究狭义相对论意义下面的引力理论,需要先研究狭义相对论之中的加速参考系的性质【而非仅仅是洛伦兹变换相连的平直的惯性参考系的性质】

\bigskip
我们首先复习了狭义相对论的基本概念【Minkowski时空;洛伦兹变换;能动量四矢量;自由粒子的运动;Action】;然后研究了匀加速运动在狭义相对论性质下的结果,我们发现匀加速的的物体自身的参考系是一个叫Rindler Coordinate的参考系,这个也是一个Minkowski时空的参考系。

\bigskip
最重要的是,由于我们认为都是Minkowski时空。我们的线元的大小应该是绝对不变的。那么这个时空下就会有一个特殊的Metric度规矩阵。我们知道,对于一个加速的参考系,我们可以通过metric来描述参考系的性质!!

\bigskip
我们进一步意识到,根据等效原理,引力也应该是用一样的方式【固定线元,考虑一些特殊的度规】进行描述的!!我们首先研究了general coordinate transformation在Minkowski时空下面的情况。我们发现一个很显然但是重要的事实:4加速度虽然是洛伦兹4矢量,但是对于一般的坐标变换并不是4矢量!!

\bigskip
我们进一步研究加速度怎么在一般坐标变换下面进行变换的,发现其按照张量变换之外还需要多加一个量也就是Christoffel symbol!!并且这个量和metric的导数是有关系的!!并且加入Chris Symbol之后就变成了一个general的tensor!!!

\bigskip
最后一步我们研修了一些Minkowski时空以及其他弯曲时空下面坐标系的例子。


\subsection{章节问题捏}

\textbf{Q1:指标的书写规则是什么?}

\textbf{指标量的书写:}
所有的指标都是有前后顺序的,因为指标的存在意味着可能的张量积的基的存在【注意是可能的,只有张量才是一定】。由于张量积是不能改变顺序的,所以我们指标都是有前后顺序的,不论是上指标还是下指标。

\textbf{指标的求和规则:}指标求和需要满足两个规则,满足这两个规则的指标自动求和:
\begin{enumerate}
  \item 指标一个上一个下
  \item 指标字母相同
\end{enumerate}

\textbf{指标约定俗成书写记号:}
对于一些有指标的量,我们存在一些约定俗成的记号写法:
\begin{enumerate}
  \item 物理量由于其协变关系,我们会固定是上指标还是下指标;当然我们也可以通过度规张量升降指标,modify物理量改变其协变关系。我们一般约定通过度规张量升降变换前后物理量使用同一个字母表示。但是物理上,$ x^\mu, A^\mu, \partial_\mu$ 这些指标都是固定的。
  \item 对于度规张量,我们认为度规张量都是下指标的。上指标的度规张量我们定义为下指标度规张量的逆矩阵。我们不难证明度规张量的逆矩阵正好按照Jacobi的逆进行变换。
  \item Jacobi矩阵我们定义$ \tensor{J}{^\mu_a} = \displaystyle\frac{\partial x^\mu}{\partial y^a} $我们一般用不同语言的字母表示不同的参考系。对于Jacobi的逆矩阵我们一般用同样的符号进行表示,但是我们知道是逆变换的矩阵:$ J^{-1 a}{}_\mu =  \tensor{J}{^a_\mu} = \displaystyle\frac{\partial y^a}{\partial x^\mu} $并满足: $ \tensor{J}{^\mu_a}\tensor{J}{^a_\nu} = \tensor{\delta}{^\mu_\nu} $ 
  \item 我们把指标形式写成矩阵形式的时候常常遇见需要写一个转置。量的转置之后和之前是两个量,所以应该使用两个记号进行标记。转置也就是把两个基的前后张量积顺序进行交换,不改变其协变性质。也就是说,转置前后的的量满足:
    $ \tensor{Tt}{_\nu^\mu} = \tensor{T}{^\mu_\nu} $  
    
    \item 我们考虑逆矩阵和升降指标的关系。对于一个坐标变换会有:
    \begin{equation}
      \tensor{\Lambda}{^a_\mu}\tensor{\Lambda}{^a_\mu} g_{ab} = g_{\mu\nu} 
      \label{eq:generalcoordinatetrans}
    \end{equation}
    对于这样的变换我们会发现升降指标之后的张量和逆矩阵满足关系:
    \begin{equation}
      \Lambda^{-1 \sigma}{}_c = \Lambda_c{}^\sigma 
      \label{eq:liftandinverse}
    \end{equation}
    由于我们的Jacobi的逆矩阵定义为:$ J^{-1 a}{}_\mu =  \tensor{J}{^a_\mu} $所以也可以有一个等号。但是,我们不希望这样计算,太混淆了!!!
\end{enumerate}

\bigskip
\textbf{Q2:对于两个指标的量「不一定是张量」我们怎么翻译成为矩阵形式?}

指标的方程写作矩阵形式除了有的时候手动 or 输入电脑进行计算没有其他意义。


\textbf{Q3: 洛伦兹变换矩阵能否合理的升降指标,记号会不会有问题?}


\section{Metric Geometry and Geodesic}

\subsection{章节内容 take home message}



\subsection{章节问题捏}



