% main.tex - 科研项目学习笔记模板
% !TeX root = main.tex
%%%%%%%%%%%%%%%%%%%%%%%%%%%%%% DOCUMENT 
\documentclass[12pt]{article}

%%%%%%%%%%%%%%%%%%%%%%%%%%%%%% PACKAGES

% 中文支持(XeLaTeX 编译)
\usepackage[UTF8]{ctex}
\usepackage{xeCJKfntef} 

% \setCJKmainfont{HanziPen SC}
% \setmainfont{HanziPen SC}


% 页面设置
\usepackage[a4paper, left=20mm, right=20mm, top=15mm, bottom=15mm]{geometry}

\PassOptionsToPackage{dvipsnames,svgnames,x11names}{xcolor}
\usepackage{xcolor}


% 数学环境及符号
\usepackage{amsmath, amssymb, amsfonts, amsthm,amsopn}
\usepackage{tensor}              % 张量指标管理
\usepackage{mathtools}           % amsmath增强
\usepackage{physics}             % 物理公式快捷命令
             % Dirac符号
\usepackage{bbold}               % 数学黑体
\usepackage{dsfont}              % 另一种数字体
\usepackage[mathscr]{eucal}     % 花体字母
\usepackage{tensor}              % 张量指标管理
\usepackage{simpler-wick}       % Wick记号
\usepackage{mathrsfs}            % 另一种花体字母

% 颜色与图形相关
\usepackage{graphicx}           % 插图支持
\usepackage{float}              % 浮动体控制
\usepackage{tikz}               % 绘图库
\usetikzlibrary{math}           % tikz数学扩展
\usepackage{geometry}
% 表格与列表
\usepackage{makecell}           % 表格多行换行
\usepackage{multicol}           % 多栏排版
\usepackage{colortbl}           % 表格颜色
\usepackage{enumitem}           % 列表自定义

% 其他辅助
\usepackage{framed}             % 有边框环境
\usepackage{tcolorbox}          % 灵活盒子环境
\tcbuselibrary{breakable}       % 盒子内容分页
\usepackage{thmtools}           % 定理环境管理
\usepackage{thm-restate}        % 定理重述
\usepackage{showlabels}         % 显示标签,调试用(完成后可注释)
\usepackage[normalem]{ulem}     % 下划线、删除线
\usepackage{hyperref}           % 超链接(最后加载)
\usepackage{cleveref}           % 智能引用(紧跟hyperref)
\usepackage{soul}

% 自定义宏包
\usepackage{macros}

% 一个中文可以高亮的包
\usepackage{cjkhl}
\definecolor{lightblue}{rgb}{.8,.8,1}

%%%%%%%%%%%%%%%%%%%%%%%%%%%%%% 自定义命令
\newcommand{\tml}{Teichmüller space}
\newcommand{\hil}{Hilbert space}
\newcommand{\mtc}{Modular Tensor Category}

%%%%%%%%%%%%%%%%%%%%%%%%%%%%%% BEGINNING OF THE DOCUMENT

\begin{document}

\title{\boldmath Lie Algebra in Particle Physics Learning Note}
\author{X. D. H.}

\maketitle

\begin{abstract}
  这是我在EPFL第一年physics project期间阅读Lie Algebra in Particle Physics一书的读书笔记。在读书笔记之外也包含了每一次讨论的习题结果!
\end{abstract}

\tableofcontents
\section{Week 1 Reading}\label{sec:Chapter1Reading} % (fold)
This is the note for chapter 1 reading.

% section Chapter1Reading (end)
\newpage
\section{Week 2 Reading}\label{sec:Week 2 Reading} % (fold)
\subsection{Take home messages for Week 2}
Note for week 2 reading of "Lie Algebras in Particl Physics" by Howard Georgi.

\hlr{对称群以及其表示}

\imp{对称群的记号定义}{
  注意!定义是【position】的变化而不是number的变化。如果是number的变化那么$ (1,2)(2,3) \neq (1,2,3) $ 所以「严格的」应该理解成【位置的变换】。也就是$ (1,2) $是吧第一个和第二个位置上面的东西进行交换而不是把数字1和2进行交换。
}

\hlr{连续的群结构}





\subsection{Explanation and Questions for Week 2}

\textbf{Question 1: 本书之中我们经常使用 $ \ket{i} $ 作为一个表示的基,这到底是什么意思?}

我觉得本书之中这样的使用其实意思就是$ \ket{i} $的意思就是$ (0,...,1,...0) $ 其中1出现在第i个位置上面。因为算符 $ D_a(g) $ 作用在这个基上面给出的矩阵就是这样的 $ \bra{i} D_a(g) \ket{j} = G $

\imp{本书对于李代数讨论的数学缺陷}{
  我们说明了连续群的生成元,根据群的乘法关系,如果写作exp parametrization的形式需要存在commutative algebra的结构。但是并没有证明这个代数结构具有代数应该有的数学性质。

  但是自然的结果应该是存在这样的性质的!!
}

\textbf{Question 2: 生成元本身自带哪些数学结构?(sec:2.2)}

请注意我们的生成元的定义是:
\begin{equation}
  X_a\equiv-i\left.\frac{\partial}{\partial\alpha_a}D(\alpha)\right|_{\alpha=0}
  \label{eq:generatordef}
\end{equation}
也就是选定某一个对于群表示的参数化之后,生成元就是表示对于这个参数的导数。并且为了方便讨论,我们一般使用指数参数化的形式。也就是我们使用这样的一组$ \{ \alpha_a\} $进行参数化,使得$ e $附近的群元素可以写作:$ D(\alpha) = exp(i \alpha_a X_a) $。

\begin{enumerate}
  \item 加法/数乘结构: 
所有的生成元都是线性映射【因为表示是这么定义的】,某两个线性空间之间的线性映射其实构成了一个线性空间,赋予「数乘+加法」结构。根据tayler expansion的性质,我们知道这样组合的生成元仍然是生成元。
\item 乘法结构:
同时,因为我们考虑的表示一般是一个线性空间自己到自己的线性映射。所以,生成元一般可以自然赋予乘法的结构。但是我们并不知道这样的乘法结构之后的物体是否仍然是一个生成元。
\item 对易结构:
只是commutation algebra的结构告诉我们,如果$ X_a $是选定某个指数参数化之后的生成元,那么$ [X_a, X_b] $也是一个对于指数参数化生成元。
\end{enumerate}

\textbf{Question 3: 请告诉我一些线性代数的基本知识球球了!!}

\textbf{实对称矩阵:} 可以进行对角化,并且所有本征值都是实数。但是正负并不确定。显然是Hermite的。

\textbf{纯虚反对称矩阵:}显然也是Hermite的,并且本征值都是纯虚数。

\textbf{Question 4: Adjoint representation 到底由什么决定?}

虽然这本书之中我们讨论的都是连续群的表示和代数的表示。但是 Adjoint representation 是由代数和群乘法结构决定的。这个和任何表示无关,这是一个由代数本身结构决定的自然的表示。这个表示选择的基就是 代数 这个线性空间!!


\imp{抽象的李代数定义以及本书之中的定义}{
  本书之中李代数定义为连续群表示的生成元(单位元附近的导数)。所以自然是表示空间自己到自己的线性映射空间。

  但是,李代数的定义是不依赖于表示的!虽然我们可以通过表示和生成元定义拥有李代数结构的线性映射构成的线性空间。但是李代数结构本身与表示无关并且可以通过抽象的方式进行定义。

  Adjoint Representation并不是由表示决定的而是李代数的抽象结构决定的。
}





% section Week 2 Reading (end)

\newpage
\section{Scratch book}\label{sec:Scratch book} % (fold)
\sout{这里会放一些写的很混沌,但懒得扔掉的东西呜呜呜呜!!!}


% section Scratch book (end)

\end{document}
