\subsection{Take home messages}
\hlr{为什么QFT是唯一能够reconcile GR和QM的理论?}

\begin{enumerate}
  \item 我们可以把Schrodinger方程写作一个平移不变的Clein-Gordon方程。但是问题是这个方程有负能解。我们可以手动消去负能量:
    \begin{equation}
      (i\hbar\partial_t-\sqrt{m^2c^4-\hbar^2c^2\nabla^2})\psi(t,x)=0\mathrm{~.}
      \label{eq:nonegativenergyspectrum}
    \end{equation}
    但是会有问题,相对论告诉我们光锥内外是不会相互影响的。但是我们可以计算上面方程的传播子:
    \begin{equation}
      A(x\to y,t)=e^{-\frac{mc}{\hbar}\sqrt{(x-y)^2-c^2t^2}}f(x-y,t)
      \label{eq:propergagtiontest}
    \end{equation}
    所以粒子有一定概率可以超过光速,这是违背相对论的。
  \item 
\end{enumerate}

\subsection{Questions and thoughts}
\textbf{Q1: 流形上面的函数是怎么定义协变的方法的?}




