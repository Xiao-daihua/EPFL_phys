\subsection{Take home messages}
\hlr{为什么QFT是唯一能够reconcile GR和QM的理论?}

\begin{enumerate}
  \item 我们可以把Schrodinger方程写作一个平移不变的Clein-Gordon方程。但是问题是这个方程有负能解。我们可以手动消去负能量:
    \begin{equation}
      (i\hbar\partial_t-\sqrt{m^2c^4-\hbar^2c^2\nabla^2})\psi(t,x)=0\mathrm{~.}
      \label{eq:nonegativenergyspectrum}
    \end{equation}
    但是会有问题,相对论告诉我们光锥内外是不会相互影响的。但是我们可以计算上面方程的传播子:
    \begin{equation}
      A(x\to y,t)=e^{-\frac{mc}{\hbar}\sqrt{(x-y)^2-c^2t^2}}f(x-y,t)
      \label{eq:propergagtiontest}
    \end{equation}
    所以粒子有一定概率可以超过光速,这是违背相对论的。
  \item 
\end{enumerate}

\subsection{Questions and thoughts}
\textbf{Q1: 流形上面的函数是怎么定义协变的方法的?}



\textbf{Q2:自然单位制下,我们到底是什么意思?}

我们自然单位制下面,我们一般把所有物理量使用能量的单位进行写作。这样只是一个记号,真实用来描述世界的时候,我们还是需要通过合理的补充$ \hbar,c $,把这个数变成我们一般的单位的。

Eg 当我们说粒子的质量是1GeV的时候,我们实际上是说这个粒子的静止能量是1GeV。也就是$ E=mc^2=1GeV $。或者说粒子的质量是 $ m = 1GeV/c^2 $。

对于电磁场的自然单位制,我们一般把$ \epsilon_0,\mu_0 $都设置为1。这样我们就有$ c = 1/\sqrt{\epsilon_0\mu_0} = 1 $。但是涉及电荷的单位的时候我们还需要注意,除了需要乘回来$ \epsilon_0,\mu_0 $之外,我们需要乘回来一个$ \sqrt{4\pi} $。

因为真正的电荷$ q_\text{真实}=q_\text{自然单位}\cdot\sqrt{4\pi\epsilon_0\hbar c}. $ 这是我们规定的。对于磁场和电场也是需要的!

\imp{自然单位制的正确用法}{
  对于自然单位制的使用,就是我们一切使用同样的能量单位进行描述。最后换算成为真实单位的时候需要把$ \hbar,c,\epsilon_0,\mu_0, 4 \pi$都补充回来,并且使用我们想要的单位制进行书写。
}

\textbf{Q3: Gauss 单位制到底是在干什么,我们在rescale什么?}

如果我们使用高斯单位制进行计算,这个时候我们仿佛只rescale了一个物理量。而不是自然单位制一样rescale所有的除了能量之外的物理量。我们rescale的其实是:。。。。



