% main.tex - 科研项目学习笔记模板
% !TeX root = main.tex
%%%%%%%%%%%%%%%%%%%%%%%%%%%%%% DOCUMENT 
\documentclass[12pt]{report}

%%%%%%%%%%%%%%%%%%%%%%%%%%%%%% PACKAGES

% 中文支持(XeLaTeX 编译)
\usepackage[UTF8]{ctex}
\usepackage{xeCJKfntef} 

% \setCJKmainfont{HanziPen SC}
% \setmainfont{HanziPen SC}


% 页面设置
\usepackage[a4paper, left=20mm, right=20mm, top=15mm, bottom=15mm]{geometry}

\PassOptionsToPackage{dvipsnames,svgnames,x11names}{xcolor}
\usepackage{xcolor}


% 数学环境及符号
\usepackage{amsmath, amssymb, amsfonts, amsthm,amsopn}
\usepackage{tensor}              % 张量指标管理
\usepackage{mathtools}           % amsmath增强
\usepackage{physics}             % 物理公式快捷命令
             % Dirac符号
\usepackage{bbold}               % 数学黑体
\usepackage{dsfont}              % 另一种数字体
\usepackage[mathscr]{eucal}     % 花体字母
\usepackage{tensor}              % 张量指标管理
\usepackage{simpler-wick}       % Wick记号
\usepackage{mathrsfs}            % 另一种花体字母

% 颜色与图形相关
\usepackage{graphicx}           % 插图支持
\usepackage{float}              % 浮动体控制
\usepackage{tikz}               % 绘图库
\usetikzlibrary{math}           % tikz数学扩展
\usepackage{geometry}
% 表格与列表
\usepackage{makecell}           % 表格多行换行
\usepackage{multicol}           % 多栏排版
\usepackage{colortbl}           % 表格颜色
\usepackage{enumitem}           % 列表自定义

% 其他辅助
\usepackage{framed}             % 有边框环境
\usepackage{tcolorbox}          % 灵活盒子环境
\tcbuselibrary{breakable}       % 盒子内容分页
\usepackage{thmtools}           % 定理环境管理
\usepackage{thm-restate}        % 定理重述
\usepackage{showlabels}         % 显示标签,调试用(完成后可注释)
\usepackage[normalem]{ulem}     % 下划线、删除线
\usepackage{hyperref}           % 超链接(最后加载)
\usepackage{cleveref}           % 智能引用(紧跟hyperref)
\usepackage{soul}

% 自定义宏包
\usepackage{macros}

% 一个中文可以高亮的包
\usepackage{cjkhl}
\definecolor{lightblue}{rgb}{.8,.8,1}

%%%%%%%%%%%%%%%%%%%%%%%%%%%%%% 自定义命令
\newcommand{\tml}{Teichmüller space}
\newcommand{\hil}{Hilbert space}
\newcommand{\mtc}{Modular Tensor Category}

%%%%%%%%%%%%%%%%%%%%%%%%%%%%%% BEGINNING OF THE DOCUMENT

\begin{document}

\title{\boldmath Modern approach Quantum Gravity note}
\author{X. D. H.}

\maketitle

\begin{abstract}
  这个笔记是2025年秋天liuy旁听Monica的Modern approach to Quantum Gravity课程时做的笔记。由于此时liuy物理水平还有限,所以需要课下补充很多preliminary的东西,也比较乱。

  由于课程风格是上课讲的比较随意。很多细节推导需要课后补充。所以笔记中会补充很多很多很多的内容。基本以monica的授课作为主线很多参考书进行补充捏。
\end{abstract}

\tableofcontents

\chapter{Lecture 1}
\subsection{Take home messages}
\hlr{为什么QFT是唯一能够reconcile GR和QM的理论?}

\begin{enumerate}
  \item 我们可以把Schrodinger方程写作一个平移不变的Clein-Gordon方程。但是问题是这个方程有负能解。我们可以手动消去负能量:
    \begin{equation}
      (i\hbar\partial_t-\sqrt{m^2c^4-\hbar^2c^2\nabla^2})\psi(t,x)=0\mathrm{~.}
      \label{eq:nonegativenergyspectrum}
    \end{equation}
    但是会有问题,相对论告诉我们光锥内外是不会相互影响的。但是我们可以计算上面方程的传播子:
    \begin{equation}
      A(x\to y,t)=e^{-\frac{mc}{\hbar}\sqrt{(x-y)^2-c^2t^2}}f(x-y,t)
      \label{eq:propergagtiontest}
    \end{equation}
    所以粒子有一定概率可以超过光速,这是违背相对论的。
  \item 
\end{enumerate}

\subsection{Questions and thoughts}
\textbf{Q1: 流形上面的函数是怎么定义协变的方法的?}




 

\chapter{Scratch Book}
\input{sb.tex}

\end{document}
